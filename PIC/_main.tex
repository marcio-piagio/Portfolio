% Options for packages loaded elsewhere
\PassOptionsToPackage{unicode}{hyperref}
\PassOptionsToPackage{hyphens}{url}
%
\documentclass[
]{book}
\usepackage{amsmath,amssymb}
\usepackage{iftex}
\ifPDFTeX
  \usepackage[T1]{fontenc}
  \usepackage[utf8]{inputenc}
  \usepackage{textcomp} % provide euro and other symbols
\else % if luatex or xetex
  \usepackage{unicode-math} % this also loads fontspec
  \defaultfontfeatures{Scale=MatchLowercase}
  \defaultfontfeatures[\rmfamily]{Ligatures=TeX,Scale=1}
\fi
\usepackage{lmodern}
\ifPDFTeX\else
  % xetex/luatex font selection
\fi
% Use upquote if available, for straight quotes in verbatim environments
\IfFileExists{upquote.sty}{\usepackage{upquote}}{}
\IfFileExists{microtype.sty}{% use microtype if available
  \usepackage[]{microtype}
  \UseMicrotypeSet[protrusion]{basicmath} % disable protrusion for tt fonts
}{}
\makeatletter
\@ifundefined{KOMAClassName}{% if non-KOMA class
  \IfFileExists{parskip.sty}{%
    \usepackage{parskip}
  }{% else
    \setlength{\parindent}{0pt}
    \setlength{\parskip}{6pt plus 2pt minus 1pt}}
}{% if KOMA class
  \KOMAoptions{parskip=half}}
\makeatother
\usepackage{xcolor}
\usepackage{longtable,booktabs,array}
\usepackage{calc} % for calculating minipage widths
% Correct order of tables after \paragraph or \subparagraph
\usepackage{etoolbox}
\makeatletter
\patchcmd\longtable{\par}{\if@noskipsec\mbox{}\fi\par}{}{}
\makeatother
% Allow footnotes in longtable head/foot
\IfFileExists{footnotehyper.sty}{\usepackage{footnotehyper}}{\usepackage{footnote}}
\makesavenoteenv{longtable}
\usepackage{graphicx}
\makeatletter
\def\maxwidth{\ifdim\Gin@nat@width>\linewidth\linewidth\else\Gin@nat@width\fi}
\def\maxheight{\ifdim\Gin@nat@height>\textheight\textheight\else\Gin@nat@height\fi}
\makeatother
% Scale images if necessary, so that they will not overflow the page
% margins by default, and it is still possible to overwrite the defaults
% using explicit options in \includegraphics[width, height, ...]{}
\setkeys{Gin}{width=\maxwidth,height=\maxheight,keepaspectratio}
% Set default figure placement to htbp
\makeatletter
\def\fps@figure{htbp}
\makeatother
\setlength{\emergencystretch}{3em} % prevent overfull lines
\providecommand{\tightlist}{%
  \setlength{\itemsep}{0pt}\setlength{\parskip}{0pt}}
\setcounter{secnumdepth}{5}
\ifLuaTeX
\usepackage[bidi=basic]{babel}
\else
\usepackage[bidi=default]{babel}
\fi
\babelprovide[main,import]{brazilian}
% get rid of language-specific shorthands (see #6817):
\let\LanguageShortHands\languageshorthands
\def\languageshorthands#1{}
\usepackage{booktabs}
\ifLuaTeX
  \usepackage{selnolig}  % disable illegal ligatures
\fi
\usepackage[]{natbib}
\bibliographystyle{plainnat}
\IfFileExists{bookmark.sty}{\usepackage{bookmark}}{\usepackage{hyperref}}
\IfFileExists{xurl.sty}{\usepackage{xurl}}{} % add URL line breaks if available
\urlstyle{same}
\hypersetup{
  pdftitle={ESTIMAÇÃO POR MÁXIMA VEROSSIMILHANÇA DO TAMANHO POPULACIONAL EM MODELOS DE CAPTURA-RECAPTURA},
  pdflang={pt-br},
  hidelinks,
  pdfcreator={LaTeX via pandoc}}

\title{ESTIMAÇÃO POR MÁXIMA VEROSSIMILHANÇA DO TAMANHO POPULACIONAL EM MODELOS DE CAPTURA-RECAPTURA}
\author{}
\date{\vspace{-2.5em}`Maringá, 30 de Agosto de 2022}

\begin{document}
\maketitle

{
\setcounter{tocdepth}{1}
\tableofcontents
}
\hypertarget{apresentauxe7uxe3o}{%
\chapter{Apresentação}\label{apresentauxe7uxe3o}}

\textbf{UNIVERSIDADE ESTADUAL DE MARINGÁ}

\textbf{PROGRAMA DE INICIAÇÃO CIENTÍFICA - PIC}

\textbf{DEPARTAMENTO DE ESTATÍSTICA}

\textbf{Orientador:} Prof.~Dr.~George Lucas Moraes Pezzott

\textbf{Acadêmico:} Márcio Roger Piagio

\textbf{Acadêmico:} Raphael Amaral Luiz

Relatório contendo os resultados finais do projeto de iniciação científica vinculado ao Programa PIC-UEM

\hypertarget{resumo}{%
\section{Resumo}\label{resumo}}

O presente projeto apresenta dois modelos de captura-recaptura: o modelo \(M_t\), que considera probabilidades diferentes no instante de captura dos animais; e o modelo \(M_{tb}\) que leva em consideração possíveis diferenças entre as probabilidades de captura e recaptura. O processo inferencial foi definido por obter as estimativas de máxima verossimilhança dos parâmetros dos modelos e os resultados numéricos deste trabalho se basearam em duas aplicações com dados reais de captura-recaptura de ratos e um estudo de simulação em diferentes cenários para avaliar a performance dos estimadores de máxima verossimilhança. No geral, foi possível concluir que o método de captura-recaptura é extremamente válido como uma técnica de amostragem para se estimar o tamanho da população e também é robusto quando se tem pelo menos 60 \% da população observada na amostra pois seus estimadores apresentam baixo viés e erro quadrático médio.

\textbf{Palavras-chave:} modelo de captura-recaptura, heterogeneidade; efeito comportamental; estimativa de máxima verossimilhançaca

\hypertarget{introduuxe7uxe3o}{%
\chapter{Introdução}\label{introduuxe7uxe3o}}

Diversas áreas do conhecimento buscam conhecer o número de elementos em uma dada população.
Na ecologia, por exemplo, o monitoramento do número de animais em uma determinada região é de suma importância no estudo e conservação da espécie \citet{seber1982},\citet{mccrea2014}.
Na área da saúde, é necessário avaliar a quantidade de indivíduos com uma determinada característica (doença, usuários de droga, \ldots) em uma cidade para promover políticas públicas e sociais \citet{bird2018multiple},\citet{bohning2020estimating}.
Por outro lado, a confiabilidade de um \textbf{software} está relacionada ao número de erros (falhas) que ele apresenta \citet{basu2001bayesian}.

Entretanto, na maioria das aplicações é impraticável observar todos os elementos
da população devido à dificuldade operacional como, dentre outros fatores, populações esquivas, excessivo tempo para execução ou alto custo financeiro.
Nestes casos, procedimentos estatísticos inferenciais são necessários para obter estimativas para
os parâmetros populacionais e o método de captura-recaptura (CR) mostra-se
uma técnica de amostragem útil e robusta.

Resumidamente, o método de captura-recaptura consiste em selecionar elementos desta população em diferentes ocasiões de amostragem.
Na primeira ocasião, uma amostra é retirada, os elementos capturados recebem uma marca e, em seguida, todos são devolvidos à população.
Após um certo período de tempo, é selecionada uma segunda amostra e realizada a contagem dos elementos marcados (recapturas), e aqueles não marcados recebem uma marca, e todos são devolvidos à população.
Este procedimento é repetido em \(k\) \((k\geq2)\) ocasiões de amostragem, e em cada ocasião é realizada a contagem do número de elementos selecionados e daqueles previamente marcados, feita a marcação dos elementos não marcados e todos são devolvidos à população.
No final do processo faz-se a inferência sobre os parâmetros populacionais baseada no número de elementos capturados e recapturados \citet{otis1978},\citet{rodrigues1988bayesian},\citet{oliveira2007},\citet{salasar2011eliminaccao},\citet{wang2015}\}.

A utilização da técnica de captura-recaptura para estimar o tamanho de uma população pode ser melhor compreendida com um exemplo simples. Considere que \({N}\) é o tamanho da população a ser estimada e \(n_1\) é o total de animais capturados na primeira época de captura, sendo todos marcados e devolvidos à população. Logo, nesta população, temos uma proporção \(n_1/N\) de animais marcados. Considere que na segunda época de captura tenhamos \(n_2\) animais capturados, dos quais \(m\) estejam marcados. A ideia é estimar a proporção de marcados da população (\(n_1/N\)) pela proporção de marcados na segunda amostra (\(m/n_2\)), isto é,
\[\widehat{\left(\dfrac{n_1}{N}\right)} = \dfrac{m}{n_2}\]
onde, resolvendo-se em \(N\), tem-se um estimador \(\hat{N}\) para o tamanho populacional, dado por

\[\hat{N} = \dfrac{n_1 n_2}{m}\]

Na literatura, este estimador é conhecido como estimador de Lincoln-Petersen, em referência aos primeiros pesquisadores a empregarem este método na ecologia, o dinamarquês \citet{PETERSEN1896}, em seu estudo sobre o fluxo migratório de peixes do mar Báltico e \citet{lincoln1930}, ao estimar o tamanho da população de patos selvagens na América do Norte. Contudo, este método foi proposto inicialmente por \citet{LAPLACE1783} para estimar o tamanho da população francesa.

Atualmente, diversos modelos de captura-recaptura são encontrados na literatura para as mais diversas aplicações. Embora tenha-se maior volume de trabalhos com aplicação na ecologia para a estimação de abundâncias de populações animais \citet{mccrea2014},\citet{royle2013}, o método pode ser aplicado em outras áreas do conhecimento. Citamos o trabalho de \citet{polonsky2021} como um exemplo de aplicação na epidemiologia, onde se buscou estimar a prevalência de Ebola via integridade do rastreamento de contatos durante o surto na República Democrática do Congo, entre os anos de 2018 e 2020. Nas àreas de políticas públicas e sociais, \citet{ryngelblum2021} empregaram um estudo sobre a análise da qualidade dos dados das mortes cometidas por policiais no município de São Paulo, Brasil, entre os anos de 2014 e 2015. Outro exemplo é um estudo onde foi possível identificar a má conduta das instituições financeiras e seus funcionários no Reino Unido entre 2004 e 2016 apresentado por \citet{ashton2021}.

\hypertarget{objetivos}{%
\section{Objetivos}\label{objetivos}}

O método de captura-recaptura vêm sendo aplicado nas mais diversas áreas do conhecimento e, por isso, o objetivo principal deste projeto é estudar essa metodologia de amostragem e a modelagem estatística para os dados provenientes desta técnica.
Com relação à modelagem, temos os seguintes objetivos específicos para o projeto:

\begin{itemize}
\item
  estudar dois tradicionais modelos estatísticos de captura-recaptura da literatura: \(M_t\) e \(M_{tb}\) \citet{otis1978};
\item
  discutir métodos de estimação dos parâmetros desses modelos e aplicá-los em dados reais da literatura para exemplificar a metodologia;
\item
  apresentar um estudo de simulação para avaliar a performance dos estimadores.
\end{itemize}

\hypertarget{desenvolvimento-materiais-e-muxe9todos}{%
\chapter{Desenvolvimento (Materiais e Métodos)}\label{desenvolvimento-materiais-e-muxe9todos}}

Considere uma população com \(N\) indivíduos. Neste projeto, supomos que a população é ``fechada'' (demográfica e geograficamente), isto é, não há mortes (saídas ou perdas) nem nascimentos (entradas) de indivíduos na população ao longo do processo de captura-recaptura. Em outras palavras, assumimos que a população de estudo é composta pelos mesmos \(N\) indivíduos desde o início da primeira ocasião de captura até o final da última.

O processo de captura-marcação-recaptura
de animais da população é aplicado da maneira tradicional: em cada ocasião de amostragem, observa-se um número aleatório (não fixado previamente) de animais desta população, registra-se o número de animais marcados e não marcados na amostra, marcando todos os não marcados e devolve-se todos (marcados e não marcados) à população. O processo é repetido em \(k\) ocasiões de amostragem. Supomos que não há animais marcados na população antes da primeira ocasião de amostragem, e que os animais não perdem suas marcas durante o processo.

Diante do exposto, definimos

\begin{itemize}
\item
  \(u_j\): número de animais não marcados na \(j\)-ésima ocasião de amostragem, \(j = 1, 2, . . . , k;\) com \(\mathbf{u}=(u_1, u_2, . . . , u_k)\)
\item
  \(m_j\) : número de animais marcados (recapturas) na \(j\)-ésima ocasião de amostragem, \(j = 2, . . . , k\) com \(\mathbf{m} = (m_2, m_3, . . . , m_k)\).
\item
  \(n_j\): número de animais selecionados (marcados e não marcados) na \(j\)-ésima ocasião de amostragem, isto é, \(n_j = u_j + m_j, ~j = 1 , 2, . . . , k;\)
\item
  \(M_j:\) número de animais marcados na população imediatamente antes da \(j\)-ésima ocasião de amostragem, \(j = 1, . . . , k,\) com \(M_1 = 0\) e \(M_{j+1} = M_j + u_j\), para \(j =2, 3, . . . , k.\)
\end{itemize}

A seguir, apresentamos um exemplo.

\emph{Suponha uma população de tamanho \(N = 100\). De fato, temos \(M_1 = 0\) pois não temos animais marcados na população antes da primeira época de captura.}

\textbf{Primeira ocasião:} \emph{Suponha que capturamos \(n_1 = 10\) animais. Consequentemente, temos \(u_1=10\) animais não marcados na amostra e um total de \(m_1=0\) animais marcados observados, isto é, nenhuma recaptura na primeira ocasião. Marcamos esses 10 animais e devolvemos à população. Observa-se que agora temos \(M_2 = M_1+u_1\) = \(0+10 = 10\) animais marcados na população.}

\textbf{Segunda ocasião:} \emph{Após certo tempo para que os animais devolvidos na primeira ocasião possam se misturar aos demais, colhemos uma segunda amostra. Suponha que agora observamos \(n_2 = 15\) animais, dos quais \(u_2=12\) não são marcados e \(m_2=3\) são marcados, isto é temos três animais recapturados. Então marca-se os 12 não marcados, e todos são devolvidos à população. Neste momento, temos \(M_3 = 10+12=22\) animais marcados na população.}

\textbf{Terceira ocasião:} \emph{Na terceira ocasião de captura, suponha um total de \(n_3=17\) animais capturados, sendo \(u_3 = 8\) animais não marcados e \(m_3 = 9\) marcados (observa-se que a marcação não permite identificar se o animal recapturado foi marcado na primeira ou segunda época de captura, embora poderia realizar uma marcação específica em cada animal para registrar tal histórico). Por fim, marcando os 8 animais e devolvendo todos à população, temos \(M_3=22 + 8 = 30\) animais observados até o momento.}

As quantidades estão apresentadas na Tabela\ref{tab:nice-tab} abaixo.

\begin{longtable}[]{@{}ccccc@{}}
\caption{\label{tab:nice-tab}Exemplo de quantidades observadas em um processo de captura-recaptura.}\tabularnewline
\toprule\noalign{}
& j & 1 & 2 & 3 \\
\midrule\noalign{}
\endfirsthead
\toprule\noalign{}
& j & 1 & 2 & 3 \\
\midrule\noalign{}
\endhead
\bottomrule\noalign{}
\endlastfoot
Animais marcados antes da ocasião & M & 0 & 10 & 22 \\
Animais capturados (não marcados) & u & 10 & 12 & 8 \\
Animais recapturados (marcados) & m & 0 & 3 & 9 \\
Animais selecionados & n & 10 & 15 & 17 \\
\end{longtable}

\emph{Em três épocas de captura, foi possível observar 30 animais distintos, de um total de \(N=100\) animais. O presente exemplo se encerra comentando que o processo de captura-recaptura poderia ser estendido para mais épocas de captura.}

A seguir, apresentamos dois modelos estatísticos para estimação do tamanho populacional utilizando as informações de capturas definidas acima.

\hypertarget{modelo-de-captura-recaptura-com-heterogeneidade}{%
\section{\texorpdfstring{Modelo de captura-recaptura com heterogeneidade}{Modelo de captura-recaptura  com heterogeneidade}}\label{modelo-de-captura-recaptura-com-heterogeneidade}}

Nesta seção vamos apresentar o modelo de captura-recaptura com heterogeneidade temporal das probabilidades de captura, comumente denotado por \(M_t\). Definimos \(p_j\) sendo a probabilidade de um animal (marcado ou não) ser capturado na \(j\)-ésima ocasião de amostragem, \(j = 1 , 2, . . . ,k\) com \(\mathbf{p}= (p_1, p_2, . . . , p_k)\).

Supomos que os animais comportam-se independente uns dos outros, isto é, a captura (ou não) de
um animal não altera a probabilidade de captura de qualquer outro. Adicionalmente, supomos que a captura (ou não) de um animal não altera sua probabilidade de recaptura, ou seja, animais marcados e não marcados tem a mesma probabilidade \(p_j\) de serem capturados na ocasião \(j\).

Note que, por se tratar de uma sequência de sucessos ou fracassos (captura o não captura) e independência das capturas e mesma probabilidade de sucesso (captura) para as quantidades \(u_j\) e \(m_j\), temos as seguintes distribuições:
\begin{equation} 
u_j|N,p_j, M_j \sim \mathrm{Binomial}(N-M_j,p_j), ~~ j = 1,2,...,k  
\end{equation}

e
\begin{equation} 
m_j|p_j,M_j \sim \mathrm{Binomial}(M_j, p_j),   ~~   j = 2,...,k  
\end{equation}

Logo, a função de verossimilhança do modelo \(M_t\) para os parâmetros \(N\) e \(\mathbf{p} = (p_1,\ldots,p_k)\), dada uma amostra de captura-marcação-recaptura \(\mathbf{u}=(u_1,u_2,\ldots,u_k)\) e \(\mathbf{m} = (m_2,m_3,\ldots,m_k)\), é definida por
\begin{align*}
L(N,\textbf{p}|\textbf{u,m}) &= \mathrm{p}(\textbf{u,m}|N, \textbf{p})\\
 &= p(u_1|N,p_1)\prod^k_{j=2}p(u_j|N,p_j,M_j)p(m_j|M_j,p_j)\\
 &= \prod^k_{j=1}\binom{N-M_j}{u_j}p^{u_j}_j(1-p_j)^{N-M_j-u_j}\times \prod^k_{j=2}\binom{M_j}{m_j}p^{m_j}_j(1-p_j)^{M_j-m_j}\\
 &= \prod^k_{j=2}\binom{M_j}{m_j}\times \prod^k_{j=1}\binom{N-M_j}{n_j-M_j}p_j^{n_j}(1-p_j)^{N-n_j}
 \end{align*}
para \(N\geq r\), onde \(r = M_k+u_k\) é o número de animais distintos observados no estudo, e \(0<p_j<1\), para \(j=1,2,,\ldots,k\), sendo \(n_j = u_j+m_j\).

Observe que, como \(M_{j+1} = M_j+u_j\), temos
\begin{align*}
  \prod^k_{j=1}\binom{N-M_j}{u_j} &= \prod^k_{j=1}\frac{(N-M_j)!}{u_j!(N-M_j - u_j)!} \\
 &= \prod_{j=1}^k\dfrac{1}{u_j!}\prod_{j=1}^k\dfrac{(N-M_j)!}{(N-M_{j+1})!}\\
 &= \prod^k_{j=1}\dfrac{1}{u_j!}\times \dfrac{N!}{(N-r)!}
\end{align*}

Logo, a função de verossimilhança pode ser reescrita em termos proporcionais à
\begin{align*}
L(N,\textbf{p}|\textbf{u,m}) &\propto  \dfrac{N!}{(N-r)!}\prod^{k}_{j=1}p_j^{n_j}(1-p_j)^{N-n_j}
\end{align*}
para \(N\geq r\) e \(0<p_j<1\), para \(j=1,2,,\ldots,k\),

\hypertarget{modelo-de-captura-recaptura-com-heterogeneidade-e-efeito-uxe0-marcauxe7uxe3o}{%
\section{\texorpdfstring{Modelo de captura-recaptura com heterogeneidade e efeito à marcação}{Modelo de captura-recaptura com   heterogeneidade e efeito à marcação}}\label{modelo-de-captura-recaptura-com-heterogeneidade-e-efeito-uxe0-marcauxe7uxe3o}}

  \bibliography{book.bib,packages.bib}

\end{document}
